%%%%%%%%%% INSTALLER LATEX %%%%%%%%%%%%
% Ce fichier produit un document qui vous explique rapidement c'est quoi LaTeX
% Et comment l'installer sur votre ordinateur.
% Ou vous pouvez simplement utiliser Overleaf...

% ======== PRÉAMBULE ================
% Ne touchez pas au préambule à moins de savoir ce que vous faites...

\documentclass[11pt]{amsart}

\usepackage[utf8]{inputenc}
\usepackage[T1]{fontenc}

\usepackage[french]{babel}

\usepackage{amsmath}
\usepackage{amsthm}
\usepackage{amsfonts}
\usepackage{amscd}
\usepackage{amssymb}
\usepackage{amstext}
\usepackage{hyperref}

\usepackage{lipsum}

\usepackage{enumerate}
\usepackage{array}
\usepackage{bbm}

\usepackage[margin=1.25in]{geometry}

% Hyperref style
\hypersetup{
	colorlinks = true
}

% Définition des environnements.
\newtheorem{theoreme}{Théorème}
\newtheorem*{theoreme*}{Théorème}
\newtheorem{proposition}{Proposition}
\newtheorem*{proposition*}{Proposition}
\newtheorem{lemme}{Lemme}[section]
\newtheorem*{lemme*}{Lemme}
\newtheorem{corollaire}{Corollaire}
\newtheorem*{corollaire*}{Corollaire}
\newtheorem{conjecture}{Conjecture}
\newtheorem*{conjecture*}{Conjecture}

\theoremstyle{definition}
\newtheorem{definition}{Définition}
\newtheorem*{definition*}{Définition}
\newtheorem{notation}{Notation}
\newtheorem*{notation*}{Notation}
\newtheorem{exercice}{Exercice}
\newtheorem*{exercice*}{Exercice}
\newtheorem{probleme}{Problème}
\newtheorem*{probleme*}{Problème}
\newtheorem{question}{Question}
\newtheorem*{question*}{Question}
\newtheorem*{solution}{Solution}

\theoremstyle{remark}
\newtheorem{exemple}{Exemple}[section]
\newtheorem*{exemple*}{Exemple}
\newtheorem{remarque}{Remarque}[section]
\newtheorem*{remarque*}{Remarque}
\newtheorem*{enigme}{Énigme}

\numberwithin{equation}{section}

% Notations utiles
\newcommand{\tq} {\text{t.q.}}

% Notations -- ensembles
\newcommand{\N} {\mathbb{N}}
\newcommand{\Z} {\mathbb{Z}}
\newcommand{\Q} {\mathbb{Q}}
\newcommand{\R} {\mathbb{R}}
\newcommand{\C} {\mathbb{C}}
\newcommand{\K} {\mathbb{K}}

\newcommand{\abs}[1]{\left\vert #1 \right\vert}
\newcommand{\card}[1]{\abs{#1}}

% Notation -- proba
\newcommand{\PR} {\mathbb P}
\newcommand{\ES} {\mathbb E}
\newcommand{\Var} {\mathrm{Var}}
\newcommand{\Cov} {\mathrm{Cov}}

% Notation -- délimiteurs :
\newcommand{\ac}[1]{\left\{ #1 \right\}} % accolades
\newcommand{\pr}[1]{\left( #1 \right)} % parenthèses
\newcommand{\ct}[1]{\left[ #1 \right]} % crochets
\newcommand{\norm}[1]{\left\Vert #1 \right\Vert} % norme
\newcommand{\floor}[1]{\left\lfloor #1 \right\rfloor} % partie entière
\newcommand{\ceil}[1]{\left\lceil #1 \right\rceil} % partie entière + 1
\newcommand{\cond}{\ \middle\vert \ }

% Différence symétrique.
\newcommand \dsym {\mathop{}\!\mathbin\bigtriangleup\mathop{}\!}

% Fonction indicatrice.
\newcommand \1 {\mathbbm 1}

%Une ligne horizontale
\newcommand \lh {
	\setlength\parindent{0pt} 
	\rule{\textwidth}{0.5pt} 
	\setlength{\parindent}{11pt} 
}
% ============== FIN DU PRÉAMBULE =========================

% ============== INFORMATIONS IMPORTANTES =================
% Vous pouvez modifier ces informations. 

% Le titre. Entre [crochets], la version écourtée à utiliser pour les entête des autres pages.
%           Entre {accolades}, la version longue. On utilise les \\ doubles barres obliques inversées \\
%           Pour changer de ligne.
\title  [Gabarit]
        {
            MAT1720 -- Introduction aux probabilités -- H20 \\
            Introduction à \LaTeX{} : concept et installation.
        }

% Le nom de l'auteur/rice.
\author {
            Thomas Davignon
        }

% La date. La commande "\today" permet d'inscrire la date du jour où le document est compilé.
\date{\today}

% ============ FIN DES INFORMATIONS =======================

% ============ DÉBUT DU DOCUMENT =========================
% C'est ici qu'on rédige le document.

% D'abord la commande \begin{document} ouvre l'environnement "document";
% C'est dans l'environnement "document" qu'on va écrire tout notre document.

\begin{document}
\maketitle

    Alors comme ça vous souhaitez apprendre à utiliser \LaTeX{} pour produire de jolis documents ?
    Le présent document (produit avec \LaTeX{}) vise à vous présenter brièvement le concept,
    ainsi qu'à vous montrer comment installer \LaTeX{} sur votre ordinateur.
    Si vous ne voulez pas installer \LaTeX{} sur votre ordinateur, vous pouvez aussi utiliser
    \href{https://fr.overleaf.com/}{Overleaf}, une application intégrée web qui 
    permet d'utiliser \LaTeX{} à travers son fureteur, sans avoir à installer
    quoi que ce soit.
    
    \section{\LaTeX{}, c'est quoi ?}
    
    \LaTeX{}, c'est un système de typographie informatique, comme \textit{Word}, de Microsoft
    Office, ou \textit{Writer}, de la suite LibreOffice.
    
    Là où \LaTeX{} est un peu particulier, c'est qu'il n'est pas conçu pour 
    vous montrer en temps réel le document sur lequel vous travaillez. À la place, \LaTeX{} fonctionne un peu plus sur le modèle d'un compilateur
    en programmation.
    
    La rédaction de documents avec \LaTeX{} se fait en deux étapes.
    
    \subsection{La rédaction du code source.}
    À cette étape, vous devez écrire dans un fichier texte ordinaire
    (avec l'extension \verb+.tex+)
    votre code source.
    
    Le code source est l'entièreté du code qui va générer votre document.
    C'est donc tout votre texte qui se retrouve dedans, ainsi que des instructions
    qui disent au compilateur \LaTeX{} comment afficher les différents éléments
    de votre document.
    
    La structure d'un document simple ressemble à ça :
    \begin{figure}[h]
    \begin{center}
        \begin{verbatim}
            \documentclass[11pt]{article}
            ...
            
            \title {Titre du document}
            \author {Nom de l'auteur/rice}
            ...
            
            \begin{document}
                \maketitle
                ...
                
            \end{document}
        \end{verbatim}
    \end{center}
    \caption{Structure-type d'un fichier source pour un document \LaTeX{} simple}
    \label{fig:structure}
    \end{figure}
    
    On va analyser ça dans l'ordre.
    \begin{enumerate}
        \item \verb+\documentclass[11pt]{article}+ \\
        \verb+...+ \\ \\
        C'est le préambule du document, qui donne toutes les instructions à LaTeX
        pour s'initialiser comme il faut.
        La commande \verb+\documentclass+ sert à indiquer à \LaTeX{}
        quel type de document on va rédiger.
        Une fois que c'est fait, on peut rajouter toutes sortes de commandes,
        pour dire à LaTeX de charger des fonctionnalités spéciales,
        pour définir nos propres commandes, etc. \\
        
        \item \verb+\title {Titre du document}+ \\
        \verb+\author {Nom de l'auteur/rice}+ \\
        \verb+...+ \\ \\
        Tout dépendant du type de document qu'on a spécifié à \LaTeX{}
        on va maintenant fournir à \LaTeX{} des informations importantes
        pour que celui-ci puisse faire la présentation do document
        correctement. Souvent, les informations incluent
        le titre, le (ou les) auteur/rice(s), mais il peut aussi
        y avoir des informations sur la date, un résumé, etc. \\
        
        \item \verb+\begin{document}+\\
        \verb+  \maketitle+ \\
        \verb+  ...+ \\
        \verb+\end{document}+ \\ \\
        C'est entre \verb+\begin{document}+ et \verb+\end{document}+ qu'on va placer
        l'entièreté du document.
        La commande \verb+\maketitle+ est souvent employée au début
        pour indique à \LaTeX{} qu'il faut générer l'en-tête du document, en utilisant
        les informations fournies juste avant. 
    \end{enumerate}
    
    La rédaction du code source peut être faite avec n'importe quel éditeur fichiers
    au format "texte" -- bloc-notes, vim, emacs, Notepad++, etc.
    
    \subsection{La compilation du document}
    Une fois que le code source est rédigé, il faut le compiler pour obtenir un 
    document PDF. C'est là que \LaTeX{} fait tout son travail : il va lire le code
    source, l'interpréter, puis créer un document PDF en suivant les instructions
    dans le code source.
    
    Pour pouvoir compiler un document avec \LaTeX{}, il faut disposer d'un \textbf{compilateur},
    et d'une distribution de \LaTeX{} qui comprend les \textit{packages} appropriés.
    
    Ça a l'air très compliqué comme ça, mais heureusement, des gens ont développé
    des outils pour rendre le travail plus facile pour nous.
    
    Dans les prochaines sections, on va voir les différentes méthodes pour
    s'installer et travailler avec \LaTeX{}.
    
    \newpage
    \section{Écrire en \LaTeX{} sous Windows}
    
    \subsection{La distribution}
    Si, comme la majorité des gens, vous utilisez un système d'exploitation de
    la famille Windows, vous allez devoir commencer par installer
    une distribution de \LaTeX{}
    
    Je recommande d'utiliser \href{https://miktex.org/}{Mik\TeX}. Sur leur site,
    dans la section \textit{Downloads}, naviguez à l'onglet \textit{Windows},
    puis \textit{Installer}. Téléchargez l'installeur, et installez toutes
    les composantes du logiciel. Mik\TeX vient avec un paneau de contrôle très pratique pour gérer les mises à jour et l'installation de nouveaux packages.
    
    \subsection{Les éditeurs}
    Comme mentionné plus haut, vous pourriez rédiger votre code source
    dans n'importe quel éditeur de texte ordinaire comme le bloc-notes
    ou Notepad++.
    
    Toutefois, il est toujours préférable d'utiliser un éditeur spécialisé, qui intègre
    les fonctionnalités pratiques comme un bouton "Compiler" qui se charge d'envoyer
    les commandes de compilation, un aperçu du résultat après la compilation,
    un menu qui montre la structure de notre document en sections et en sous-sections,
    etc.
    
    Ici il existe une panoplie d'options, et vous pouvez choisir celle qui vous plaît.
    Moi j'aime bien \href{https://www.xm1math.net/texmaker/index_fr.html}{\TeX Maker}.
    L'interface ressemble à celle de \href{https://fr.overleaf.com/}{Overleaf}.
    
    \section{Écrire en \LaTeX{} sous MacOS}
    
    \subsection{La distribution}
    Encore une fois, il vous faudra une distribution et un éditeur.
    Pour la distribution, \href{https://tug.org/mactex/}{Mac\TeX} semble
    être assez largement utilisée.
    
    \subsection{Les éditeurs}
    Et encore une fois c'est l'embarras du choix. \href{https://www.xm1math.net/texmaker/index_fr.html}{\TeX Maker} est aussi disponible
    pour MacOS.
    
    \section{Écrire en \LaTeX{} sous Linux}
    
    \subsection{La distribution}
    La plupart des distributions Linux viennent déjà équipées avec une distribution \LaTeX{};
    toutefois celle-ci n'est peut-être pas à jour. Vous pouvez rechercher la distribution \TeX Live sur votre dépôt de paquets favori.
    
    \subsection{Les éditeurs}
    Vous pouvez choisir celui que vous voulez, mais \href{https://www.xm1math.net/texmaker/index_fr.html}{\TeX Maker} est aussi disponible
    sous la majorité des distributions Linux.
    
    \section{Overleaf}
    
    Vous pouvez utiliser \LaTeX{} sans installer quoi que ce soit sur votre ordinateur
    (sauf un fureteur pour naviguer sur Internet). C'est possible grâce à
    \href{https://fr.overleaf.com/}{Overleaf}.
    
    Vous devrez vous créer un compte; je suggère que vous utilisiez votre adresse courriel
    institutionnel de l'UdeM.
    Une fois que c'est fait, vous pouvez créer des projets directement en-ligne.
    
    Tous vos projets sont sauvegardés dans le nuage, et vous y avez accès facilement,
    partout où vous avez accès à l'internet. Vous pouvez également télécharger les fichiers
    \verb+.tex+ pour les éditer hors-ligne, télécharger les PDF finis, etc.
    
    \href{https://fr.overleaf.com/}{Overleaf} propose aussi des tas de ressources
    didactiques et de la documentation très utile pour apprendre à utiliser \LaTeX{}.
    
    Je vous recommande fortement de l'essayer.
    
    Un projet overleaf contenant entre autres ce document-ci est disponible
    à l'adresse :
    \url{https://fr.overleaf.com/read/xfpmgrgrjhnj}
    
\end{document}