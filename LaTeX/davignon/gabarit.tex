%%%%%%%%% GABARIT LaTEX %%%%%%%%%%%%%%
% Vous pouvez utiliser ce fichier comme gabarit de départ pour vos travaux.

% ======== PRÉAMBULE ================
% Ne touchez pas au préambule à moins de savoir ce que vous faites...

\documentclass[11pt]{amsart}

\usepackage[utf8]{inputenc}
\usepackage[T1]{fontenc}

\usepackage[french]{babel}

\usepackage{amsmath}
\usepackage{amsthm}
\usepackage{amsfonts}
\usepackage{amscd}
\usepackage{amssymb}
\usepackage{amstext}
\usepackage{hyperref}

\usepackage{lipsum}

\usepackage{enumerate}
\usepackage{array}
\usepackage{bbm}

\usepackage[margin=1.25in]{geometry}

% Hyperref style
\hypersetup{
	colorlinks = true
}

% Définition des environnements.
\newtheorem{theoreme}{Théorème}
\newtheorem*{theoreme*}{Théorème}
\newtheorem{proposition}{Proposition}
\newtheorem*{proposition*}{Proposition}
\newtheorem{lemme}{Lemme}[section]
\newtheorem*{lemme*}{Lemme}
\newtheorem{corollaire}{Corollaire}
\newtheorem*{corollaire*}{Corollaire}
\newtheorem{conjecture}{Conjecture}
\newtheorem*{conjecture*}{Conjecture}

\theoremstyle{definition}
\newtheorem{definition}{Définition}
\newtheorem*{definition*}{Définition}
\newtheorem{notation}{Notation}
\newtheorem*{notation*}{Notation}
\newtheorem{exercice}{Exercice}
\newtheorem*{exercice*}{Exercice}
\newtheorem{probleme}{Problème}
\newtheorem*{probleme*}{Problème}
\newtheorem{question}{Question}
\newtheorem*{question*}{Question}
\newtheorem*{solution}{Solution}

\theoremstyle{remark}
\newtheorem{exemple}{Exemple}[section]
\newtheorem*{exemple*}{Exemple}
\newtheorem{remarque}{Remarque}[section]
\newtheorem*{remarque*}{Remarque}
\newtheorem*{enigme}{Énigme}

\numberwithin{equation}{section}

% Notations utiles
\newcommand{\tq} {\text{t.q.}}

% Notations -- ensembles
\newcommand{\N} {\mathbb{N}}
\newcommand{\Z} {\mathbb{Z}}
\newcommand{\Q} {\mathbb{Q}}
\newcommand{\R} {\mathbb{R}}
\newcommand{\C} {\mathbb{C}}
\newcommand{\K} {\mathbb{K}}

\newcommand{\abs}[1]{\left\vert #1 \right\vert}
\newcommand{\card}[1]{\abs{#1}}

% Notation -- proba
\newcommand{\PR} {\mathbb P}
\newcommand{\ES} {\mathbb E}
\newcommand{\Var} {\mathrm{Var}}
\newcommand{\Cov} {\mathrm{Cov}}

% Notation -- délimiteurs :
\newcommand{\ac}[1]{\left\{ #1 \right\}} % accolades
\newcommand{\pr}[1]{\left( #1 \right)} % parenthèses
\newcommand{\ct}[1]{\left[ #1 \right]} % crochets
\newcommand{\norm}[1]{\left\Vert #1 \right\Vert} % norme
\newcommand{\floor}[1]{\left\lfloor #1 \right\rfloor} % partie entière
\newcommand{\ceil}[1]{\left\lceil #1 \right\rceil} % partie entière + 1
\newcommand{\cond}{\ \middle\vert \ }

% Différence symétrique.
\newcommand \dsym {\mathop{}\!\mathbin\bigtriangleup\mathop{}\!}

% Fonction indicatrice.
\newcommand \1 {\mathbbm 1}

%Une ligne horizontale
\newcommand \lh {
	\setlength\parindent{0pt} 
	\rule{\textwidth}{0.5pt} 
	\setlength{\parindent}{11pt} 
}
% ============== FIN DU PRÉAMBULE =========================

% ============== INFORMATIONS IMPORTANTES =================
\title  [Travail final]
        {
            MAT1720 -- Introduction aux probabilités -- H20 \\
            Travail final
        }

\author {
            Mon Nom
        }

\date{\today}

\begin{document}
	\maketitle
	
	\lh
	\section*{Consignes pédagogiques}
	
	\begin{enumerate}[$\bullet$]
		\item Vous devez rendre votre travail \textbf{au format PDF, par StudiUM.} \\
		Je n'accepterai aucun travail rendu par une autre méthode ou dans un autre format.
		
		\item Vous devez rendre votre travail\textbf{ au plus tard le mardi 21 avril à 23h 59.} \\ 
		Ça vous donne exactement $4$ semaines. Au-delà de cette date, vous serez pénalisé.e.s de $15\%$ par jour de retard.
		
		\item \textbf{Votre travail doit être typographié. Aucun travail manuscrit ne sera accepté}. \\
		Des ressources seront disponibles en ligne pour vous aider à maîtriser le logiciel de typographie mathématique de votre choix (Microsoft Word, LibreOffice Writer, \LaTeX{}). Il y a définitivement une courbe d'apprentissage avec ces outils (spécialement avec \LaTeX{}). Toutefois, le temps consacré à cet apprentissage ne sera pas perdu. \LaTeX{} est le standard en rédaction scientifique, mais peu importe l'outil, il est impératif que vous apprenniez à typographier des mathématiques adéquatement.
		
		\item \textbf{Vous devez rendre un travail par personne.} \\
		Vous pouvez communiquer entre vous et vous entraider, bien sûr, mais vous devez remettre chacun.e votre propre travail.
	\end{enumerate}
	\lh
	\newpage
	
	\section{Le jeu \em Clue \em}
	
	\begin{probleme} On n'a pas encore distribué les cartes.
	\begin{enumerate}[(a)]
		\item Supposons que Axel devait deviner le crime maintenant. 
		Quelle serait la probabilité que son choix soit le bon ?
		
		\begin{solution}
			%% VOTRE SOLUTION ICI
		\end{solution}
		\vspace{0.3cm}
		
		\item Supposons que Axel devra deviner le crime immédiatement
		après avoir vu ses trois cartes. Est-ce que cela augmentera
		sa probabilité de deviner le crime correctement ? Expliquer pourquoi ou pourquoi pas.
		
		\begin{solution}
			%% VOTRE SOLUTION ICI
		\end{solution}
		\vspace{0.3cm}
		
		
		\item Donner la probabilité qu'Axel devinera le crime correctement
		après avoir vu ses cartes.
		
		\begin{solution}
			%% VOTRE SOLUTION ICI
		\end{solution}
		\vspace{0.3cm}
		
		
	\end{enumerate}
	\end{probleme}
	\vspace{0.5cm}
	
	\begin{probleme}
		Le crime a été choisi au hasard, on a brassé les cartes, on les a distribué.
		\begin{enumerate}[(a)]
			\item Quelle est la probabilité qu'Axel ait Miss Scarlett dans son jeu ?
		
		\begin{solution}
			%% VOTRE SOLUTION ICI
		\end{solution}
		\vspace{0.3cm}
		
			
			\item Quelle est la probabilité que Miss Scarlett soit la coupable
			sachant qu'Axel n'a pas Miss Scarlett dans son jeu ?
		
		\begin{solution}
			%% VOTRE SOLUTION ICI
		\end{solution}
		\vspace{0.3cm}
		
			
		\end{enumerate}
	\end{probleme}
	\newpage
	
	\section{L'espérance d'une variable aléatoire positive discrète}
		
	\begin{probleme}
		Soit $X$ une variable aléatoire discrète positive de support $V = \{ v_1, v_2, v_3, \dots\}$ (on assume que $0 \leq v_1 < v_2 < v_3 < \cdots$). On suppose également que $\lim_{x \to \infty} x \overline F(x) = 0$.
		
		\begin{enumerate}[(a)]
			\item Montrer que pour tout $k \in\N$, $x \in [v_k, v_{k+1})$,
			$$ \overline F (x) = \overline F(v_k),$$
			et que pour tout $x \in [0, v_1)$, $\overline F(x) = 1$.
		
		\begin{solution}
			%% VOTRE SOLUTION ICI
		\end{solution}
		\vspace{0.3cm}
		
			
			\item Déduire que
			$$\int_0^{v_1} \overline F(x) dx = v_1,\qquad \int_{v_k}^{v_{k+1}} \overline F(x) dx = \overline F(v_k) (v_{k+1} - v_k),$$
			et finalement
			$$\int_0^\infty \overline F (x) dx = v_1 + \sum_{k=1}^\infty \overline F(v_k) (v_{k+1} - v_k).$$
		
		\begin{solution}
			%% VOTRE SOLUTION ICI
		\end{solution}
		\vspace{0.3cm}
		
			
			
			\item Si $p_X (x)$ est la fonction de masse probabilités de $X$, montrer que
			$$p_X (v_{k+1}) = \overline F (v_k) - \overline F (v_{k+1}).$$
		
		\begin{solution}
			%% VOTRE SOLUTION ICI
		\end{solution}
		\vspace{0.3cm}
		
			
			
			\item En utilisant le lemme de sommation d'Abel, montrer que
			$$\ES \ct{X} = \int_0^\infty \overline F(x) dx.$$
		
		\begin{solution}
			%% VOTRE SOLUTION ICI
		\end{solution}
		\vspace{0.3cm}
		
			
			\item Montrer que si $v_k = (k-1)$ pour tout $k \in \N$ (c'est-à-dire si $X$ prend des valeurs entières non-négatives), alors
			$$\ES \ct{X} = \sum_{k=1}^\infty \PR \ac{X \geq k}.$$
		
		\begin{solution}
			%% VOTRE SOLUTION ICI
		\end{solution}
		\vspace{0.3cm}
		
		\end{enumerate}
	\end{probleme}
	\vspace{0.5cm}
	
	\newpage
	\section{Distributions conditionnelles}
	
	\begin{probleme}
		Soient $X, Y$ deux variables aléatoires indépendantes géométriques de paramètre $p$.
		\begin{enumerate}[(a)]
			\item En procédant par calcul, montrer que
			$$\PR \ac{X + Y = k} = (k-1)p^2 (1-p)^{k-2}.$$
		
		\begin{solution}
			%% VOTRE SOLUTION ICI
		\end{solution}
		\vspace{0.3cm}
		
			
			\item Montrer que la loi conditionnelle de $X$ sachant que $X+Y = k$
			est une loi équiprobable sur $\{1, 2, 3, \dots, k-1\}$.
		
		\begin{solution}
			%% VOTRE SOLUTION ICI
		\end{solution}
		\vspace{0.3cm}
		
			
		\end{enumerate}
	\end{probleme}
	\vspace{0.5cm}
	
	\begin{probleme}
		Soient $X_1, X_2, X_3, \dots$ une suite de variables aléatoires indépendantes
		et identiquement distribuées de paramètres $\lambda > 0$.
		
		On définit $S_n = \sum_{i=1}^n X_i$.
		
		\begin{enumerate}[(a)]
			\item Trouver la densité conditionnelle
			de $X_1$ sachant $S_n$.
		
		\begin{solution}
			%% VOTRE SOLUTION ICI
		\end{solution}
		\vspace{0.3cm}
		
			
			\item Montrer que sachant $S_n = t$, $X_1/t$ suit une loi Beta ($1,n-1$).
		
		\begin{solution}
			%% VOTRE SOLUTION ICI
		\end{solution}
		\vspace{0.3cm}
		
			
			\item En remarquant que la densité conditionnelle de $X_1/t$ sachant que $S_n= t$ ne dépend pas de $t$, déduire que $X_1/S_n$ suit une loi Beta($1, n-1$).
		
		\begin{solution}
			%% VOTRE SOLUTION ICI
		\end{solution}
		\vspace{0.3cm}
		
			
			\item Utiliser un argument d'interchangeabilité pour justifier que
			$X_i/S_n$ suit une loi Beta($1, n-1$) pour tout $i$ de $1$ à $n$.
		
		\begin{solution}
			%% VOTRE SOLUTION ICI
		\end{solution}
		\vspace{0.3cm}
		
			
			
			\end{enumerate}
	\end{probleme}
		
	\newpage
	
	\section{Le paradoxe de Saint-Pétersbourg}
	
	\begin{probleme}
		Supposons qu'on joue plusieurs parties consécutives et que l'enjeu remporté à la $i$ème partie est $X_i$ -- les $X_i$ sont indépendants et identiquement distribués comme $2^{Y_i-1}$, où les $Y_i$ sont une suite de variables aléatoires géométriques indépendantes de paramètre $p=1/2$.
		
		On va noter $G_n = \sum_{i=1}^n X_i - n C$ le gain net après $n$ parties.
		
		Supposons que $h$ soit l'objectif à atteindre -- c'est le montant d'argent qu'on
		veut gagner, net.
		
		\begin{enumerate}[(a)]
			\item Pour un certain paramètre $K$ entier positif (à fixer plus tard), on se définit une nouvelle suite de variables aléatoires : $$T_n = \min\{X_n, 2^K\}$$
		
		\begin{solution}
			%% VOTRE SOLUTION ICI
		\end{solution}
		\vspace{0.3cm}
		
			
			\item Expliquer pourquoi $T_n \leq X_n$.
		
		\begin{solution}
			%% VOTRE SOLUTION ICI
		\end{solution}
		\vspace{0.3cm}
		
			
			\item Trouver $\PR \ac{T_n = 2^k}$ pour $k = 0, 1, 2, \dots,K$.
		
		\begin{solution}
			%% VOTRE SOLUTION ICI
		\end{solution}
		\vspace{0.3cm}
		
			
			\item Montrer que $$\ES \ct{T_n} = \frac{K+2}{2}.$$
		
		\begin{solution}
			%% VOTRE SOLUTION ICI
		\end{solution}
		\vspace{0.3cm}
		
			
			\item On fixe maintenant $K = 2 C$. On définit $\mu = C + 1 = \ES \ct{T_1}$, et
			$$H_n = \sum_{i=1}^n T_i - n C.$$
		
			Montrer que $\ES \ct{H_n} = n$.
		
		\begin{solution}
			%% VOTRE SOLUTION ICI
		\end{solution}
		\vspace{0.3cm}
			
			
			
			\item Utiliser la loi des grands nombres faible pour montrer que $h \in \R$,
			$$\lim_{n \to \infty} \PR \ac{H_n > h} = 1$$
		
		\begin{solution}
			%% VOTRE SOLUTION ICI
		\end{solution}
		\vspace{0.3cm}
			
			\item Déduire que $$\lim_{n \to \infty} \PR\ac{G_n > h} = 1$$ pour tout $n$.
		
		\begin{solution}
			%% VOTRE SOLUTION ICI
		\end{solution}
		\vspace{0.3cm}
			
		\end{enumerate}
	\end{probleme}
	
	\newpage
	\section{Le processus branchant}
	
	\begin{probleme}
		Soit $\psi$ la fonction génératrice des probabilités de la distribution 
		marginale des $X_{n, i}$. Soit également $\psi_n$ la fonction génératrice 
		des probabilités de $Z_n$.
	
		\begin{enumerate}[(a)]
			\item Montrer que
			$$\ES \ct{s^{Z_{n+1}} \cond Z_n} = \psi (s)^{Z_n}.$$
		
		\begin{solution}
			%% VOTRE SOLUTION ICI
		\end{solution}
		\vspace{0.3cm}
		
			
			\item Déduire que
			$$\psi_{n+1} = \psi_n \circ \psi. \qquad (\text{i.e. } \psi_{n+1} (s) = \psi_n ( \psi (s)) )$$
		
		\begin{solution}
			%% VOTRE SOLUTION ICI
		\end{solution}
		\vspace{0.3cm}
			
			\item Montrer que $\ES \ct{Z_n} = \mu^n$, où $\mu = \psi'(1)$.
		
		\begin{solution}
			%% VOTRE SOLUTION ICI
		\end{solution}
		\vspace{0.3cm}
			
			
			\item On suppose que $0 < \mu < 1$.
			 Prouver que $Z_n$ converge vers $0$ presque sûrement.
			 \em Indice : \em inspirez vous de la preuve de la loi forte des grands nombres.
		
		\begin{solution}
			%% VOTRE SOLUTION ICI
		\end{solution}
		\vspace{0.3cm}
			 
			 
			 \item Étant donné que les $Z_n$ sont des variables aléatoires entières positives,
			 comment interprétez-vous ce résultat ?
		
		\begin{solution}
			%% VOTRE SOLUTION ICI
		\end{solution}
		\vspace{0.3cm}
		
	\end{enumerate}
	\end{probleme}
		
\end{document}