%%%%%%%%%%%%%%% LES BASES %%%%%%%%%%%%%%%%
% Ce fichier produit un document expliquant les bases de LaTeX.
% Évidemment, il est rédigé avec LaTeX et vous pouvez vous en servir
% pour comparer le résultat avec le code.

% ======== PRÉAMBULE ================
% Ne touchez pas au préambule à moins de savoir ce que vous faites...

\documentclass[11pt]{amsart}

\usepackage[utf8]{inputenc}
\usepackage[T1]{fontenc}

\usepackage[french]{babel}

\usepackage{amsmath}
\usepackage{amsthm}
\usepackage{amsfonts}
\usepackage{amscd}
\usepackage{amssymb}
\usepackage{amstext}
\usepackage{hyperref}

\usepackage{lipsum}

\usepackage{enumerate}
\usepackage{array}
\usepackage{bbm}

\usepackage[margin=1.25in]{geometry}

% Hyperref style
\hypersetup{
	colorlinks = true
}

% Définition des environnements.
\newtheorem{theoreme}{Théorème}
\newtheorem*{theoreme*}{Théorème}
\newtheorem{proposition}{Proposition}
\newtheorem*{proposition*}{Proposition}
\newtheorem{lemme}{Lemme}[section]
\newtheorem*{lemme*}{Lemme}
\newtheorem{corollaire}{Corollaire}
\newtheorem*{corollaire*}{Corollaire}
\newtheorem{conjecture}{Conjecture}
\newtheorem*{conjecture*}{Conjecture}

\theoremstyle{definition}
\newtheorem{definition}{Définition}
\newtheorem*{definition*}{Définition}
\newtheorem{notation}{Notation}
\newtheorem*{notation*}{Notation}
\newtheorem{exercice}{Exercice}
\newtheorem*{exercice*}{Exercice}
\newtheorem{probleme}{Problème}
\newtheorem*{probleme*}{Problème}
\newtheorem{question}{Question}
\newtheorem*{question*}{Question}
\newtheorem*{solution}{Solution}

\theoremstyle{remark}
\newtheorem{exemple}{Exemple}[section]
\newtheorem*{exemple*}{Exemple}
\newtheorem{remarque}{Remarque}[section]
\newtheorem*{remarque*}{Remarque}
\newtheorem*{enigme}{Énigme}

\numberwithin{equation}{section}

% Notations utiles
\newcommand{\tq} {\text{t.q.}}

% Notations -- ensembles
\newcommand{\N} {\mathbb{N}}
\newcommand{\Z} {\mathbb{Z}}
\newcommand{\Q} {\mathbb{Q}}
\newcommand{\R} {\mathbb{R}}
\newcommand{\C} {\mathbb{C}}
\newcommand{\K} {\mathbb{K}}

\newcommand{\abs}[1]{\left\vert #1 \right\vert}
\newcommand{\card}[1]{\abs{#1}}

% Notation -- proba
\newcommand{\PR} {\mathbb P}
\newcommand{\ES} {\mathbb E}
\newcommand{\Var} {\mathrm{Var}}
\newcommand{\Cov} {\mathrm{Cov}}

% Notation -- délimiteurs :
\newcommand{\ac}[1]{\left\{ #1 \right\}} % accolades
\newcommand{\pr}[1]{\left( #1 \right)} % parenthèses
\newcommand{\ct}[1]{\left[ #1 \right]} % crochets
\newcommand{\norm}[1]{\left\Vert #1 \right\Vert} % norme
\newcommand{\floor}[1]{\left\lfloor #1 \right\rfloor} % partie entière
\newcommand{\ceil}[1]{\left\lceil #1 \right\rceil} % partie entière + 1
\newcommand{\cond}{\ \middle\vert \ }

% Différence symétrique.
\newcommand \dsym {\mathop{}\!\mathbin\bigtriangleup\mathop{}\!}

% Fonction indicatrice.
\newcommand \1 {\mathbbm 1}

%Une ligne horizontale
\newcommand \lh {
	\setlength\parindent{0pt} 
	\rule{\textwidth}{0.5pt} 
	\setlength{\parindent}{11pt} 
}
% ============== FIN DU PRÉAMBULE =========================

% ============== INFORMATIONS IMPORTANTES =================
% Vous pouvez modifier ces informations. 

% Le titre. Entre [crochets], la version écourtée à utiliser pour les entête des autres pages.
%           Entre {accolades}, la version longue. On utilise les \\ doubles barres obliques inversées \\
%           Pour changer de ligne.
\title  [Gabarit]
        {
            MAT1720 -- Introduction aux probabilités -- H20 \\
            Les bases de \LaTeX{} 
        }

% Le nom de l'auteur/rice.
\author {
            Thomas Davignon
        }

% La date. La commande "\today" permet d'inscrire la date du jour où le document est compilé.
\date{\today}

% ============ FIN DES INFORMATIONS =======================

% ============ DÉBUT DU DOCUMENT =========================
% C'est ici qu'on rédige le document.

% D'abord la commande \begin{document} ouvre l'environnement "document";
% C'est dans l'environnement "document" qu'on va écrire tout notre document.

\begin{document}

% La commande "\maketitle" crée l'en-tête de la première page.
% Elle affiche le titre et le nom de l'auteur/rice.
\maketitle

\section{Introduction}
\label{s:Introduction}

Le présent document est un guide qui vous expliquera rapidement les bases de l'utilisation
de \LaTeX{}. Il est accompagné d'un gabarit que vous pourrez utiliser pour votre travail
final.

Dans les prochaines sections, nous verrons comment écrire du code en \LaTeX{},
mais le document n'est certainement pas complet.
Vous êtes invité.e.s à consulter les multiples tutoriels disponibles en ligne.
\href{https://fr.overleaf.com/}{Overleaf} comprend plusieurs tutoriels
pour vous aider, mais il existe aussi plein d'autres ressources.
    
Un projet overleaf contenant entre autres ce document-ci est disponible
à l'adresse :
\url{https://fr.overleaf.com/read/xfpmgrgrjhnj}

\subsection{Le texte}
\label{ss:texte}

Commençons sans plus tarder par la base de la base.
Pour rédiger du texte avec \LaTeX{}, il suffit de l'écrire,
sur une ligne de code ou sur plusieurs lignes de code, entre
les balises \verb+\begin{document}+ et \verb+\end{document}+.
Le compilateur se chargera de tout mettre en forme pour que ça soit joli,
de justifier les paragraphes, etc.
Vous remarquerez que le document commence par la commande
\verb+\maketitle+ -- cette commande sert à créer l'en-tête avec le titre.

Pour changer de paragraphe, comme je viens de le faire, il faut sauter deux lignes. 
Si on veut passer à la ligne suivante, mais \textit{sans changer de paragraphe},
on utilise les double barres obliques inversées : \verb+\\+.
\begin{exemple}
\label{ex:exempleDeParagraphe}
    Si j'écris le code suivant :
    \begin{center}
    \begin{verbatim}
        Ceci est une phrase. \\
        Ceci est une autre phrase. \\
        Un villancico est une composition poétique et musicale du Moyen 
        Âge et de la Renaissance ibérique, que l'on retrouve dans les 
        littératures espagnole (villancico) et portugaise 
        (vilancete), proche des noëls français. 
        C'est un chant qui est donné à la période de Noël.
        
        Ceci est un nouveau paragraphe. \\
        Bryan Berg (né le 21 mars 1974) est un professionnel 
        américain dans la construction de châteaux de cartes 
        (en anglais cardstacker). Architecte de formation, 
        il construit des châteaux de cartes à grande échelle et détient 
        divers records du monde dans ce domaine.
    \end{verbatim}
    \end{center}
    
    J'obtiens le résultat suivant :
    \vspace{0.2cm}
    \begin{center}
    \begin{minipage}{0.9\textwidth}
        \setlength{\parindent}{12pt}
        Ceci est une phrase. \\
        Ceci est une autre phrase. \\
        Un villancico est une composition poétique et musicale du Moyen 
        Âge et de la Renaissance ibérique, que l'on retrouve dans les 
        littératures espagnole (villancico) et portugaise 
        (vilancete), proche des noëls français. 
        C'est un chant qui est donné à la période de Noël.
        
        Ceci est un nouveau paragraphe. \\
        Bryan Berg (né le 21 mars 1974) est un professionnel 
        américain dans la construction de châteaux de cartes 
        (en anglais cardstacker). Architecte de formation, 
        il construit des châteaux de cartes à grande échelle et détient 
        divers records du monde dans ce domaine.
    \end{minipage}
    \end{center}
    \vspace{0.2cm}
    
    Remarquez où se trouvent les alinéas -- c'est là que l'on change de paragraphe.
\end{exemple}

Remarquez en outre que je peux écrire dans le code des caractères accentués. Si j'écris \verb+ Allô +, j'obtiens le résultat : Allô ! Cela fonction car j'utilise le package \verb+inputenc+ avec l'argument \verb+utf8+ (commande \verb+\usepackage[utf8]{inputenc}+ dans le préambule).

Les guillemets français sont ouverts avec la commande \verb+\og+ et fermés avec la commande \verb+\fg{}+.

\begin{exemple}
\label{ex:exempleDeGuillemets}
    Le code suivant :
    \begin{center}
    \begin{verbatim}
        \og Allô les amis !\fg{}, dit-il à ses amis lorsqu'il les vit.
    \end{verbatim}
    \end{center}
    
    Produit le résultat suivant :
    \vspace{0.2cm}
    \begin{center}
    \begin{minipage}{0.9\textwidth}
        \setlength{\parindent}{12pt}
        \og Allô les amis !\fg{}, dit-il à ses amis lorsqu'il les vit.
    \end{minipage}
    \end{center}
    \vspace{0.2cm}
\end{exemple}

\subsection{Les commentaires}
\label{ss:commentaires}

Vous pouvez écrire directement dans le code du texte qui sera ignoré par le
compilateur LaTeX. En fait, sur chaque ligne, tout ce qui suit un signe de pourcentage
\verb+%+ est ignoré par LaTeX. Ça s'appelle \og les commentaires \fg{}
et c'est très utile pour s'aider à se repérer dans le code.

Si on veut insérer un signe de pourcentage dans le texte,
il faut \og l'échapper \fg{} en le précédent d'une barre oblique inversée (backslash) comme ceci : \verb+\%+.

\begin{exemple}
\label{ex:commentaires}
    Le code suivant :
    \begin{center}
        \begin{verbatim}
            % Ça c'est un commentaire.
            Ça c'est pas un commentaire, % mais ça oui !
            et ça c'est pas un commentaire non plus.
            J'espère que vous avez 100\% compris !
        \end{verbatim}
    \end{center}
    
    Produit le résultat suivant :
    \vspace{0.2cm}
    \begin{center}
        \begin{minipage}{0.9\textwidth}
            \setlength{\parindent}{12pt}
            % Ça c'est un commentaire.
            Ça c'est pas un commentaire, % mais ça oui !
            et ça c'est pas un commentaire non plus.
            J'espère que vous avez 100\% compris !
        \end{minipage}
    \end{center}
\end{exemple}

On va maintenant créer une nouvelle section.

\section{Les sections, les sous-sections et les références}
\label{s:sectionsEtSousSections}

Pour créer une nouvelle section, on utilise la commande \\ \verb+\section{Titre de la section}+, \\ et on met entre accollades le titre qu'on souhaite donner à notre section.

Si on veut créer une sous-section, on peut utiliser la commande \\ \verb+\subsection{Titre de la sous-section}+. \\
Comme ceci.

\subsection{Créer une sous-sous-section}
\label{ss:sousSousSections}

On peut créer une sous-sous-section en employant la commande \\ \verb+\subsubsection{Titre de la sous-sous-section}+. \\
Comme ceci.

\subsubsection{Les labels et les références}
\label{sss:labelsEtReferences}

Dans \LaTeX{}, on peut créer des \og étiquettes \fg{} attachées à différents objets -- sections, sous-sections, environnements, etc.

Pour créer une étiquette, il suffit d'employer la commande \\
\verb+\label{nom_de_l_etiquette}+. \\
Si on veut ensuite référer à l'objet ainsi étiqueté par son numéro dans le texte, il suffit d'employer la commande \\
\verb+\ref{nom_de_l_etiquette}+, \\
et \LaTeX{} va automatiquement nous fournir le numéro. C'est très pratique si on change des choses; on n'a pas à se souvenir de quel numéro est associé à quoi.

\begin{exemple}
\label{ex:exempleDeReference}
    Dans le présent document, le code, on avait
    \begin{center}
    \begin{verbatim}
        ...
        \begin{exemple}
        \label{ex:exempleDeGuillemets}
            ...
        \end{exemple}
        ...
    \end{verbatim}
    \end{center}
    
    Ainsi, le code suivant :
    \begin{center}
    \begin{verbatim}
        L'exemple sur les guillemets est l'exemple \ref{ex:exempleDeGuillemets}.
    \end{verbatim}
    \end{center}
    
    Produit le résultat suivant :
    \vspace{0.2cm}
    \begin{center}
    \begin{minipage}{0.9\textwidth}
        \setlength{\parindent}{12pt}
        L'exemple sur les guillemets est l'exemple \ref{ex:exempleDeGuillemets}.
    \end{minipage}
    \end{center}
    \vspace{0.2cm}
\end{exemple}

\begin{remarque*}
    Vous remarquerez que le numéro de l'exemple produit par la commande \verb+\ref+ est coloré en rouge,
    et produit un lien vers l'exemple.
    C'est parce qu'on a chargé le package \verb+hyperref+ (commande \verb+\usepackage{hyperref}+ 
    dans le préambule).
\end{remarque*}

\begin{exemple}
\label{ex:exempleDeReferencesDansUneSection}
    Dans le code du présent document, on avait
    \begin{center}
    \begin{verbatim}
        ...
        \section{Les sections, les sous-sections et les références}
        \label{s:sectionsEtSousSections}
        ...
    \end{verbatim}
    \end{center}
    
    Ainsi, le code suivant :
    \begin{center}
    \begin{verbatim}
        La section \ref{s:sectionsEtSousSections} porte sur 
        les sections et les sous-sections.
    \end{verbatim}
    \end{center}
    
    Produit le résultat suivant :
    \vspace{0.2cm}
    \begin{center}
    \begin{minipage}{0.9\textwidth}
        \setlength{\parindent}{12pt}
        La section \ref{s:sectionsEtSousSections} porte sur 
        les sections et les sous-sections.
    \end{minipage}
    \end{center}
    \vspace{0.2cm}
\end{exemple}

\section{Les mathématiques}

Il existe deux types d'environnements mathématiques en \LaTeX{} :

\subsection{L'environnement de mathématiques de type \og inline \fg{}.}
\label{ss:inline}

Cet environnement permet d'écrire des mathématiques à même le texte.
Il est délimité par des signes de dollar.

\begin{exemple}
\label{ex:exempleDeMathsInline}
    Le code suivant :
\begin{center}
\begin{verbatim}
    Soit $x$ une variable et $f(x)$ une fonction continue 
    qui dépend de $x$.
    Alors, on peut définir une courbe dans le plan
    par l'équation $y=f(x)$.
\end{verbatim}
\end{center}

Produit le résultat suivant :
\vspace{0.2cm}
\begin{center}
\begin{minipage}{0.9\textwidth}
    \setlength{\parindent}{12pt}
    Soit $x$ une variable et $f(x)$ une fonction continue 
    qui dépend de $x$.
    Alors, on peut définir une courbe dans le plan
    par l'équation $y=f(x)$.
\end{minipage}
\end{center}
\vspace{0.2cm}
\end{exemple}

Notez que le mode \og inline \fg{} affichera de façon différente les grands symboles dans
le but de garder la hauteur des lignes la plus uniforme possible.
\begin{exemple}
\label{ex:inlineTaille}
    Le code suivant :
    \begin{center}
        \begin{verbatim}
            Ceci est une formule inline : $\sum_{i=1}^\infty r^{i-1}
                                     = \frac{1}{1-r}$.
        \end{verbatim}
    \end{center}
    
    Produit le résultat suivant :
    \vspace{0.2cm}
    \begin{center}
        \begin{minipage}{0.9\textwidth}
        \setlength{\parindent}{12pt}
            Ceci est une formule inline : $\sum_{i=1}^\infty r^{i-1}
                                     = \frac{1}{1-r}$.
        \end{minipage}
    \end{center}
    \vspace{0.2cm}
    Notez comment les bornes de la somme sont tassées sur le côté, la fraction est plus petite,
    etc.
\end{exemple}
    
\subsection{Les environnement de mathématiques de type \og display \fg{}.}
\label{ss:display}

Il existe plusieurs types d'environnements de mathématiques qui vont mettre nos expressions
en évidence dans notre document.
    
\subsubsection{Le plus simple :  \texttt{\$\$...\$\$}.}
\label{sss:signeDePiasse}

Si on délimite simplement notre code mathématique par des doubles-signes de dollar,
\LaTeX{} va afficher le code en mode \og display \fg{}.

\begin{exemple}
\label{ex:exempleDeMathsDisplay}
    Le code suivant :
    \begin{center}
    \begin{verbatim}
        Soit $x$ une variable et $f(x)$ une fonction continue 
        qui dépend de $x$.
        Alors, on peut définir une courbe dans le plan
        par l'équation $$y=f(x).$$
    \end{verbatim}
    \end{center}
    
    Produit le résultat suivant :
    \vspace{0.2cm}
    \begin{center}
    \begin{minipage}{0.9\textwidth}
        \setlength{\parindent}{12pt}
        Soit $x$ une variable et $f(x)$ une fonction continue 
        qui dépend de $x$.
        Alors, on peut définir une courbe dans le plan
        par l'équation $$y=f(x).$$
    \end{minipage}
    \end{center}
    \vspace{0.2cm}
\end{exemple}
        
\subsubsection{Les environnements \texttt{equation} et \texttt{equation*}.}
\label{sss:equation}

Le package \verb+amsmath+ nous fournit des environnements un peu plus spécialisés pour écrire
des mathématiques.

Par exemple, l'environnement \verb+equation+ produit une équation en mode \og display \fg{}, 
comme \verb+$$+...\verb+$$+, mais avec un petit numéro. Et si on utilise la commande 
\verb+\label+, on peut y référer plus tard avec la commande \verb+\eqref+.

\begin{exemple}
\label{ex:exempleDeMathsEquation}
    Le code suivant :
    \begin{center}
    \begin{verbatim}
        Soit $x$ une variable et $f(x)$ une fonction continue 
        qui dépend de $x$.
        Alors, on peut définir une courbe dans le plan
        par l'équation 
        \begin{equation}
        \label{eq:equationCourbe}
            y = f(x).
        \end{equation}
        
        On voit par l'équation \eqref{eq:equationCourbe} que ...
    \end{verbatim}
    \end{center}
    
    Produit le résultat suivant :
    \vspace{0.2cm}
    \begin{center}
    \begin{minipage}{0.9\textwidth}
        \setlength{\parindent}{12pt}
        Soit $x$ une variable et $f(x)$ une fonction continue 
        qui dépend de $x$.
        Alors, on peut définir une courbe dans le plan
        par l'équation 
        \begin{equation}
        \label{eq:equationCourbe}
            y = f(x).
        \end{equation}
        
        On voit par l'équation \eqref{eq:equationCourbe} que ...
    \end{minipage}
    \end{center}
    \vspace{0.2cm}
\end{exemple}

On peut également utiliser l'environnement \verb+equation*+ si on ne veut pas
que l'équation ait de numéro. Le résultat est alors très similaire à
ce qui arrive lorsqu'on fait \verb+$$+...\verb+$$+.
        
\subsubsection{Les environnements \texttt{align} et \texttt{align*}.}
\label{sss:align}

Parfois il arrive qu'on veuille écrire des expressions sur plusieurs lignes.

Dans ces cas, on peut utiliser l'environnement \verb+align+.
Dans l'environnement \verb+align+, on change de ligne en utilisant la double barre oblique
inversée : \verb+\\+
On utilise l'esperluette : \verb+&+, pour placer des marques d'alignement sur chaque ligne.

Le résultat sera une forme de tableau, avec les colonnes délimitées par des \verb+&+ et les lignes
par des \verb+\\+.

\begin{exemple}
\label{ex:exempleDeMathsAlign}
    Le code suivant :
    \begin{center}
    \begin{verbatim}
        On considère les deux équations suivantes :
        \begin{align}
            2 y + 3 x & = -1 \label{eq:Equation1} \\
            6 y - 7 x & = 15 \label{eq:Equation2}
        \end{align}
    \end{verbatim}
    \end{center}
    
    Produit le résultat suivant :
    \vspace{0.2cm}
    \begin{center}
    \begin{minipage}{0.9\textwidth}
        \setlength{\parindent}{12pt}
        On considère les deux équations suivantes :
        \begin{align}
            2 y + 3 x & = -1 \label{eq:Equation1} \\
            6 y - 7 x & = 15 \label{eq:Equation2}
        \end{align}
    \end{minipage}
    \end{center}
    \vspace{0.2cm}
\end{exemple}

On remarque que, puisqu'il y aura un numéro par ligne, on doit mettre une étiquette
par ligne avec la commande \verb+\label+.

Si on ne veut pas étiqueter numéroter toutes les lignes, on peut utiliser la commande
\verb+\nonumber+ sur les lignes qu'on ne veut pas numéroter.

\begin{exemple}
\label{ex:exempleDeMathsNonumber}
    Le code suivant :
    \begin{center}
    \begin{verbatim}
        On a que
        \begin{align}
            \int_0^1 x^n dx &= \left[ \frac{x^{n+1}}{n+1} \right]_0^1 
                                \nonumber \\
                            &= \frac{1}{n+1} - \frac{0}{n+1}
                                \nonumber \\
                            &= \frac{1}{n+1} 
                                \label{eq:IntXN}
        \end{align}
    \end{verbatim}
    \end{center}
    
    Produit le résultat suivant :
    \vspace{0.2cm}
    \begin{center}
    \begin{minipage}{0.9\textwidth}
        \setlength{\parindent}{12pt}
        On a que
        \begin{align}
            \int_0^1 x^n dx &= \left[ \frac{x^{n+1}}{n+1} \right]_0^1 
                                \nonumber \\
                            &= \frac{1}{n+1} - \frac{0}{n+1}
                                \nonumber \\
                            &= \frac{1}{n+1} \label{eq:IntXN}
        \end{align}
    \end{minipage}
    \end{center}
    \vspace{0.2cm}
\end{exemple}

Finalement, si on ne veut numéroter aucune ligne, on peut employer l'environnement
\verb+align*+.

\begin{exemple}
\label{ex:exempleDeMathsAlign*}
    Le code suivant :
    \begin{center}
    \begin{verbatim}
        On a que
        \begin{align*}
            S_n &= \sum_{i=1}^n r^{i-1} \\
                &= 1 + r S_{n} - r^{n} \\
        \end{align*}
        d'où
        \begin{equation}
        \label{eq:SommeGeometriqueTronquee}
            S_n = 1 + r + r^2 + \cdots + r^{n-1} = \frac{1-r^n}{1-r}.
        \end{equation}
        
        En prenant la limite lorsque $n$ tend vers l'infini
        lorsque $\vert r \vert < 1$
        dans l'équation \eqref{eq:SommeGeometriqueTronquee}
        \begin{equation}
        \label{eq:SommeGeometrique}
            \lim_{n \to \infty} S_n = \frac{1}{1-r}.
        \end{equation}
    \end{verbatim}
    \end{center}
    
    Produit le résultat suivant :
    \vspace{0.2cm}
    \begin{center}
    \begin{minipage}{0.9\textwidth}
        \setlength{\parindent}{12pt}
        On a que
        \begin{align*}
            S_n &= \sum_{i=1}^n r^{i-1} \\
                &= 1 + r S_{n} - r^{n} \\
        \end{align*}
        d'où
        \begin{equation}
        \label{eq:SommeGeometriqueTronquee}
            S_n = 1 + r + r^2 + \cdots + r^{n-1} = \frac{1-r^n}{1-r}.
        \end{equation}
        
        En prenant la limite lorsque $n$ tend vers l'infini lorsque $\vert r \vert < 1$
        dans l'équation \eqref{eq:SommeGeometriqueTronquee}, on trouve que
        \begin{equation}
        \label{eq:SommeGeometrique}
            \lim_{n \to \infty} S_n = \frac{1}{1-r}.
        \end{equation}
    \end{minipage}
    \end{center}
    \vspace{0.2cm}
\end{exemple}

Un élément crucial à savoir concernant la typographie de formules mathématiques : les équations et expressions font partie des phrases du texte. Lorsqu'on affiche une expression en mode \og display \fg{}, il faut absolument que la ponctuation qui suive soit incluse avec l'équation.

\begin{exemple}
\label{ex:exemplePonctuationDisplay}
    Le tableau suivant indique la façon de placer la ponctuation avec une équation.
    
    \begin{table}[h] \centering
    \begin{tabular}{p{0.4\textwidth}  p{0.4\textwidth}}
        \textbf{Correct : } & \textbf{Incorrect : } \\
         \begin{minipage}{0.36\textwidth}
            \vspace{0.1cm}
            Il suffit de remarquer que
            $$a_n = \left( 1- \frac{1}{n} \right)^n,$$
            et de prendre la limite lorsque $n$ tend vers l'infini.
         \end{minipage}&
        \begin{minipage}{0.36\textwidth}
            \vspace{0.1cm}
            Il suffit de remarquer que
            $$a_n = \left( 1- \frac{1}{n} \right)^n$$
            ,et de prendre la limite lorsque $n$ tend vers l'infini.
        \end{minipage}
    \end{tabular}
    \end{table}
\end{exemple}

\subsubsection{La commande \texttt{text} }
\label{ss:text}

Si on veut, en plein milieu d'un environnement mathématique, écrire un peu de texte, on peut employer
la commande \verb+\text+.
\begin{exemple}
\label{ex:exempleDeTexte}
     Le code suivant :
    \begin{center}
    \begin{verbatim}
        Soit $$f(x) = \begin{cases}
                1 & \text{ si $x \geq 0$ } \\
                0 & \text{si $x < 0$ }.
                \end{cases} $$
    \end{verbatim}
    \end{center}
    
    Produit le résultat suivant :
    \vspace{0.2cm}
    \begin{center}
    \begin{minipage}{0.9\textwidth}
        \setlength{\parindent}{12pt}
        Soit $$f(x) = \begin{cases}
                1 & \text{ si $x \geq 0$ } \\
                0 & \text{si $x < 0$ }.
                \end{cases} $$
    \end{minipage}
    \end{center}
    \vspace{0.2cm}
\end{exemple}

\begin{remarque*}
    On remarque qu'il est tout à fait permis, à l'intérieur de l'argument de la commande \verb+\text+,
    d'ouvrir un environnement de mathématiques \og inline \fg{}.
\end{remarque*}

\subsection{Les symboles courants}
\label{ss:symboles}

Dans la section qui suit, nous présentons différents symboles couramment utilisés pour rédiger des mathématiques avec \LaTeX{}.

\subsubsection{Les lettres, les chiffres, etc.}

Dans les environnements mathématiques, on doit écrire du code \LaTeX{} mathématique. Par exemple, pour écrire des noms de variables, il suffit d'écrire les lettres correspondantes. Toutefois, il existe
des commandes qui permettent d'employer d'autres fontes. Le tableau suivant montre les commandes importantes :

\begin{table}[h]
    \centering
    \begin{tabular}{p{0.2\textwidth} c p{0.5\textwidth}}
         \verb+x+ & $x$ & La fonte ordinaire.\\
         \verb+\mathbf{x}+ & $\mathbf x$ & Les caractères gras en mathématiques. \\
         \verb+\mathrm{Hom}+ & $\mathrm{Hom}$ & Pour les noms de fonctions, comme $\sin$, $\cos$, $\tan$... \\
         \verb+\mathcal{ABC}+ & $\mathcal{ABC}$ & Les caractères calligraphiques. \\
         \verb+\mathfrak{ABC}+ & $\mathfrak{ABC}$ & Les caractères gothiques. \\
         \verb+\mathbb{R}+ & $\mathbb{R}$ & La fonte \og blackboard \fg{}, pour les ensembles, etc. \\
         \verb+\mathbbm{1}+ & $\mathbbm{1}$ & La fonte \og blackboard \fg{} par défaut ne permet pas d'afficher des chiffres. Mais c'est possible si on utilise le package \verb+bbm+ (commande \verb+\usepackage{bbm}+ dans le préambule).
    \end{tabular}
    \vspace{0.2cm}
    \caption{Les différentes fontes de caractères.}
    \label{tab:fontes}
\end{table}

Noter que tous les caractères ne sont pas disponibles dans toutes les fontes.

\LaTeX{} nous donne accès à tout un jeu de caractères supplémentaires, notamment les lettres grecques que nous aimons tou.te.s. Le tableau \ref{tab:alphabetGrec} montre les commandes \LaTeX{} à utiliser pour produire les différents caractères de l'alphabet grec, notamment les majuscules et les minuscules.

\begin{table}[htb]
    \centering
    \begin{tabular}{p{0.25\textwidth} c p{0.1\textwidth} | p{0.25\textwidth} c p {0.1\textwidth}}
         \verb+A, \alpha+           & $A, \alpha$           & alpha     &
            \verb+N, \nu+               & $N, \nu$              & nu        \\
         \verb+B, \beta+            & $B, \beta$            & beta      &
            \verb+\Xi, \xi+             & $\Xi, \xi$            & xi        \\
         \verb+\Gamma, \gamma+      & $\Gamma, \gamma$      & gamma     &
            \verb+O, o+                 & $O, o$                & omicron   \\
         \verb+\Delta, \delta+      & $\Delta, \delta$      & delta     &
            \verb+\Pi, \pi+             & $\Pi, \pi$            & pi        \\
         \verb+E, \epsilon+         & $E, \epsilon$         & epsilon   &
            \verb+P, \rho+              & $P, \rho$             & rho       \\
         \verb+Z, \zeta+            & $Z, \zeta$            & zeta      &
            \verb+\Sigma, \sigma+       & $\Sigma, \sigma$      & sigma     \\
         \verb+H, \eta+             & $H, \eta$             & eta       &
            \verb+T, \tau+              & $T, \tau$             & tau       \\
         \verb+\Theta, \theta+      & $\Theta, \theta$      & theta     &
            \verb+\Upsilon, \upsilon+   & $\Upsilon, \upsilon$  & upsilon   \\
         \verb+I, \iota+            & $I, \iota$            & iota      &
            \verb+\Phi, \phi+           & $\Phi, \phi$          & phi       \\
         \verb+K, \kappa+           & $K, \kappa$           & kappa     &
            \verb+X, \chi+              & $X, \chi$             & chi       \\
         \verb+\Lambda, \lambda+    & $\Lambda, \lambda$    & lambda    &
            \verb+\Psi, \psi+           & $\Psi, \psi$          & psi       \\
         \verb+M, \mu+              & $M, \mu$              & mu        &
            \verb+\Omega, \omega+       & $\Omega, \omega$      & omega.
    \end{tabular}
    \vspace{0.2cm}
    \caption{L'alphabet grec et les commandes \LaTeX{} correspondantes.}
    \label{tab:alphabetGrec}
\end{table}

Noter que les commandes \verb+\varepsilon+, \verb+\vartheta+, \verb+\varpi+, \verb+\varrho+, \verb+\varsigma+ et \verb+\varphi+ produisent des symboles alternatifs pour les lettres grecques epsilon, theta, pi, rho, sigma et phi, tels que présentés au tableau \ref{tab:altLettresGrecques}.

\begin{table}[htb]
    \centering
    \begin{tabular}{p{0.3\textwidth} c}
         \verb+\epsilon, \varepsilon+   & $\epsilon, \varepsilon$   \\
         \verb+\theta, \vartheta+       & $\theta, \vartheta$       \\
         \verb+\pi, \varpi+             & $\pi, \varpi$             \\
         \verb+\rho, \varrho+           & $\rho, \varrho$           \\
         \verb+\sigma, \varsigma+       & $\sigma, \varsigma$       \\
         \verb+\phi, \varphi+           & $\phi, \varphi$
    \end{tabular}
    \vspace{0.2cm}
    \caption{Versions alternatives pour certaines lettres grecques.}
    \label{tab:altLettresGrecques}
\end{table}

Le tableau \ref{tab:symboles} fournit en outre plusieurs symboles pratiques couramment utilisés.
\begin{table}[htb]
    \centering
    \begin{tabular}{r c  r c   r c}
         \verb+\emptyset+   & $\emptyset$   &
            \verb+\cup+         & $\cup$        &
                \verb+\cap+         & $\cap$        \\
        \verb+\in+          & $\in$         &
            \verb+\subseteq+    & $\subseteq$   &
                \verb+\supseteq+  & $\supseteq$ \\
        \verb+=+            & $=$           &
            \verb+\geq+         & $\geq$        &
                \verb+\leq+         & $\leq$        \\
        \verb.+.            & $+$           &
            \verb+>+            & $>$           &
                \verb+<+            &$<$            \\
        \verb+-+            & $-$           &
            \verb+\times+       & $\times$      &
                \verb+\div+         & $\div$        \\
        \verb+\Rightarrow+  & $\Rightarrow$ &
            \verb+\Leftarrow+   & $\Leftarrow$  &
                \verb+\Leftrightarrow+  & $\Leftrightarrow$  \\
        \verb+\infty+       & $\infty$      &
            \verb+\notin+       & $\notin$      &
                \verb+\neq+         & $\neq$        \\
        \verb+\setminus+    & $\setminus$   &
                                &               &
                                    &               \\
        
    \end{tabular}
    \vspace{0.2cm}
    \caption{Quelques autres symboles utiles}
    \label{tab:symboles}
\end{table}

Le tableau \ref{tab:fonctions} fournit des exemples de fonctions prédéfinies.
\begin{table}[htb]
    \centering
    \begin{tabular}{r l | r l | r l}
        \verb+\sin (x)+ & $\sin(x)$  &
            \verb+\cos (x)+ & $\cos(x)$ &
                \verb+\tan (x)+ &$\tan(x)$ \\
        \verb+\csc (x)+ & $\csc(x)$ &
            \verb+\sec (x)+ & $\sec(x)$ &
                \verb+\cot (x)+ & $\cot(x)$ \\
        \verb+\arcsin (x)+ & $\arcsin(x)$ &
            \verb+\arccos (x)+ & $\arccos(x)$ &
                \verb+\arctan (x)+ & $\arctan(x)$ \\
        \verb+\log (x)+ & $\log(x)$ &
            \verb+\ln (x)+ & $\ln(x)$ &
                \verb+\exp (x)+ & $\exp(x)$ \\
        \verb+\sqrt{x}+ & $\sqrt{x}$ &
            &   &
                & \\
    \end{tabular}
    \vspace{0.2cm}
    \caption{Tableau des raccourcis fréquents pour des fonctions}
    \label{tab:fonctions}
\end{table}

\subsection{Les constructions.}
\label{ss:grandsSymboles}

\subsubsection{Les exposants,les indices et les fractions.}
\label{sss:exposantsIndices}

Pour ajouter un indice, on emploie le tiret de soulignement: \verb+_{indice}+. Pour ajouter un exposant, on emploie l'accent circonflexe : \verb+^{exposant}+.

Noter que si l'argument est un seul caractère, ou une seule commande (sans argument), on peut se passer des \{accolades\}.
\begin{exemple}
\label{ex:indicesExposants}
     Le code suivant :
    \begin{center}
    \begin{verbatim}
        Le polynôme général de degré $n$ à une variable $x$ a la forme
        $$ p(x) = a_0 + a_1 x + a_2 x^2 + \cdots + a_{n-1}x^{n-1} + a_n x^n.$$
    \end{verbatim}
    \end{center}
    
    Produit le résultat suivant :
    \vspace{0.2cm}
    \begin{center}
    \begin{minipage}{0.9\textwidth}
        \setlength{\parindent}{12pt}
        Le polynôme général de degré $n$ à une variable $x$ a la forme
        $$ p(x) = a_0 + a_1 x + a_2 x^2 + \cdots + a_{n-1}x^{n-1} + a_n x^n.$$
    \end{minipage}
    \end{center}
    \vspace{0.2cm}
\end{exemple}

On peut construire une fraction avec la commande \verb+\frac{numerateur}{denominateur}+.
\begin{exemple}
\label{ex:fractions}
    Le code suivant :
    \begin{center}
    \begin{verbatim}
        Les fonctions rationnelles sont de la forme
        $$ f(x) = \frac{p(x)}{q(x)}.$$
    \end{verbatim}
    \end{center}
    
    Produit le résultat suivant :
    \vspace{0.2cm}
    \begin{center}
    \begin{minipage}{0.9\textwidth}
        \setlength{\parindent}{12pt}
        Les fonctions rationnelles sont de la forme
        $$ f(x) = \frac{p(x)}{q(x)}.$$
    \end{minipage}
    \end{center}
    \vspace{0.2cm}
\end{exemple}

\subsubsection{Les grands symboles}
\label{sss:grandsSymboles}

On peut écrire de grands symboles (par exemple de sommation) comme $\sum$ et $\prod$ avec les commandes \verb+\sum+ et \verb+\prod+ respectivement. Le tableau \ref{tab:grandsSymboles} montre l'usage de plusieurs de ces grands symboles. On utilise souvent les indices et les exposants pour spécifier les bornes.
On utilise aussi les indices pour spécifier les limites à prendre.

\begin{table}[htb]
    \centering
    \begin{tabular}{p{0.2\textwidth} p{0.4\textwidth} c}
        Sommation :     & \verb+\sum_{i=1}^{n} x_i+     & $\displaystyle{\sum_{i=1}^n x_i}$         \\
        Produit :       &\verb+\prod_{i=1}^{n} y_i+     & $\displaystyle{\prod_{i=1}^n y_i}$        \\
        Intégrale :     &\verb+\int_a^b f(t) dt+        & $\displaystyle{\int_a^b f(t) dt}$         \\
        Réunion :       &\verb+\bigcup_{i \geq 0} E_i+  & $\displaystyle{\bigcup_{i \geq 0} E_i}$   \\
        Limites :       &\verb+\lim_{n \to \infty} a_n+ & $\displaystyle{\lim_{n \to \infty} a_n}$  \\
        etc.            &                               &                                           \\
    \end{tabular}
    \vspace{0.2cm}
    \caption{Usage des grands symboles.}
    \label{tab:grandsSymboles}
\end{table}

\begin{exemple}
\label{ex:grandsSymboles}
    Le code suivant :
    \begin{center}
    \begin{verbatim}
        Soit $f : \mathbb R \to \mathbb R$ une fonction différentiable.
        Alors,
        $$ f(x) = f(a) + \int_a^x \lim_{h \to 0} \frac{f(t+h)-f(t)}{h} dt.$$
    \end{verbatim}
    \end{center}
    
    Produit le résultat suivant :
    \vspace{0.2cm}
    \begin{center}
    \begin{minipage}{0.9\textwidth}
        \setlength{\parindent}{12pt}
        Soit $f : \mathbb R \to \mathbb R$ une fonction différentiable.
        Alors,
        $$ f(x) = f(a) + \int_a^x \lim_{h \to 0} \frac{f(t+h)-f(t)}{h} dt.$$
    \end{minipage}
    \end{center}
    \vspace{0.2cm}
\end{exemple}

\subsubsection{Les délimiteurs}
\label{sss:delimiteurs}

\begin{exemple}
\label{ex:petitsDelimiteurs}
    Le code suivant :
    \begin{center}
    \begin{verbatim}
        Montrer que pour $\vert r \vert < 1$ :
        $$ \sum_{i=1}^\infty [\lim_{n\to\infty}(1-\frac{(i-1)\log(r)}{n})^n] $$
    \end{verbatim}
    \end{center}
    
    Produit le résultat suivant :
    \vspace{0.2cm}
    \begin{center}
    \begin{minipage}{0.9\textwidth}
        \setlength{\parindent}{12pt}
        Montrer que pour $\vert r \vert < 1$ :
        $$ \sum_{i=1}^\infty [\lim_{n\to\infty}(1-\frac{(i-1)\log(r)}{n})^n] $$
    \end{minipage}
    \end{center}
    
    C'est très laid... On voudrait que les crochets et les parenthèses s'adaptent en grandeur.
    \vspace{0.2cm}
\end{exemple}

Pour régler ce problème, \LaTeX{} nous fournit les commandes \verb+\left+, \verb+\middle+ et \verb+\right+.
On les utilise en paires (parfois en trio avec \verb+\middle+ au centre), pour indiquer à \LaTeX{} quel délimiteur redimensionner par rapport à la taille des expressions à l'intérieur.

\begin{exemple}
\label{ex:grandsDelimiteurs}
    Le code suivant :
    \begin{center}
    \begin{verbatim}
        Montrer que pour $\vert r \vert < 1$ :
        $$ \sum_{i=1}^\infty 
            \left[ \lim_{n\to\infty} 
                \left( 1-\frac{(i-1)\log(r)}{n} \right)^n
                    \right] $$
    \end{verbatim}
    \end{center}
    
    Produit le résultat suivant :
    \vspace{0.2cm}
    \begin{center}
    \begin{minipage}{0.9\textwidth}
        \setlength{\parindent}{12pt}
        Montrer que pour $\vert r \vert < 1$ :
        $$ \sum_{i=1}^\infty 
            \left[ \lim_{n\to\infty} 
                \left( 1-\frac{(i-1)\log(r)}{n} \right)^n
                    \right] $$
    \end{minipage}
    \end{center}
    C'est beaucoup mieux !
    \vspace{0.2cm}
\end{exemple}

\begin{exemple}
\label{ex:middle}
    Le code suivant :
    \begin{center}
    \begin{verbatim}
        La probabilité que nous recherchons est
        $$\mathbb P \left\{ \bigcup_{i = 1}^\infty E_i \ \middle\vert \ F \right\}$$
    \end{verbatim}
    \end{center}
    
    Produit le résultat suivant :
    \vspace{0.2cm}
    \begin{center}
    \begin{minipage}{0.9\textwidth}
        \setlength{\parindent}{12pt}
        La probabilité que nous recherchons est
        $$\mathbb P \left\{ \bigcup_{i = 1}^\infty E_i \ \middle\vert \ F \right\}$$
    \end{minipage}
    \end{center}
    
    Ici, on a utilisé \verb+\middle+ pour que la barre verticale (\verb+\vert+) soit ajustée à la taille des accolades \verb+\{ \}+.
    \vspace{0.2cm}
\end{exemple}

\subsubsection{Les matrices}
\label{sss:matrices}

On note des matrices en utilisant les environnements \verb+pmatrix+, \verb+bmatrix+ ou simplement \verb+matrix+.

\begin{exemple}
\label{ex:matrices}
    Le code suivant :
    \begin{center}
    \begin{verbatim}
        Voici trois matrices :
        $$
        \begin{matrix}
            a & b \\ c & d
        \end{matrix}
            \quad
        \begin{pmatrix}
            a & b \\ c & d
        \end{pmatrix}
            \quad
        \begin{bmatrix}
            a & b \\ c & d
        \end{bmatrix}
        $$
    \end{verbatim}
    \end{center}
    
    Produit le résultat suivant :
    \vspace{0.2cm}
    \begin{center}
    \begin{minipage}{0.9\textwidth}
        \setlength{\parindent}{12pt}
        Voici trois matrices :
        $$
        \begin{matrix}
            a & b \\ c & d
        \end{matrix}
            ,\qquad
        \begin{pmatrix}
            a & b \\ c & d
        \end{pmatrix}
            ,\qquad
        \begin{bmatrix}
            a & b \\ c & d
        \end{bmatrix}
        $$
    \end{minipage}
    \end{center}
    \vspace{0.2cm}
\end{exemple}

L'environnement \verb+matrix+ permet de choisir ses délimiteurs soi-même.

On peut utiliser l'environnement spécial \verb+cases+ pour la notation d'une fonction définie par parties.
\begin{exemple}
\label{ex:fonctionParParties}
    
    Le code suivant :
    \begin{center}
    \begin{verbatim}
        Considérons la fonction
        $$ f(x) = \begin{cases}
                f_1(x) & \text{si $x \leq \lambda$} \\
                f_2(x) & \text{si $x > \lambda$}.
                \end{cases} $$
    \end{verbatim}
    \end{center}
    
    Produit le résultat suivant :
    \vspace{0.2cm}
    \begin{center}
    \begin{minipage}{0.9\textwidth}
        \setlength{\parindent}{12pt}
        Considérons la fonction
        $$ f(x) = \begin{cases}
                f_1(x) & \text{si $x \leq \lambda$} \\
                f_2(x) & \text{si $x > \lambda$}.
                \end{cases} $$
    \end{minipage}
    \end{center}
    \vspace{0.2cm}
\end{exemple}

\section{Les environnements}

Nous avons déjà vu plusieurs exemples d'environnements.
Chaque type d'environnement a un nom, et on le crée en utilisant les commandes \verb+\begin{nomDeLEnvironnement}+ et \verb+\end{nomDeLEnvironnement}+.

Il existe plusieurs environnements particulièrement utiles pour rédiger avec \LaTeX{}

\subsection{Les environnements de listes}

Il arrive qu'on veuille créer des listes, soit à puces, soient numérotées. \LaTeX{} fournit deux environnements à ces fins : \verb+itemize+ et \verb+enumerate+.

\subsubsection{L'environnement \texttt{itemize}.}

L'environnement \verb+itemize+ sert à créer des listes à puces simples comme celle-ci :
\begin{itemize}
    \item Steak haché;
    \item Manger à chien;
    \item Scott Towels\textsuperscript{TM};
\end{itemize}

\begin{exemple}
\label{ex:exempleItemize}
Le code suivant :
    \begin{center}
    \begin{verbatim}
        Voici une liste à puces :
        \begin{itemize}
            \item Premier élément;
            \item Deuxième élément;
            \item Troisième élément;
        \end{itemize}
    \end{verbatim}
    \end{center}
    
    Produit le résultat suivant :
    \vspace{0.2cm}
    \begin{center}
    \begin{minipage}{0.9\textwidth}
        \setlength{\parindent}{12pt}
        Voici une liste à puces :
        \begin{itemize}
            \item Premier élément;
            \item Deuxième élément;
            \item Troisième élément;
        \end{itemize}
    \end{minipage}
    \end{center}
    \vspace{0.2cm}
\end{exemple}

\subsubsection{L'environnement \texttt{enumerate}.}

L'environnement \verb=enumerate= sert à créer des listes ordonnées, comme celle-ci :
\begin{enumerate}
    \item Lewis Hamilton;
    \item Valteri Bottas;
    \item Max Verstappen.
\end{enumerate}

\begin{exemple}
\label{ex:exempleEnumerate}
Le code suivant :
    \begin{center}
    \begin{verbatim}
        Voici une liste ordonnée :
        \begin{enumerate}
            \item Premier élément;
            \item Deuxième élément;
            \item Troisième élément;
        \end{enumerate}
    \end{verbatim}
    \end{center}
    
    Produit le résultat suivant :
    \vspace{0.2cm}
    \begin{center}
    \begin{minipage}{0.9\textwidth}
        \setlength{\parindent}{12pt}
        Voici une liste ordonnée :
        \begin{enumerate}
            \item Premier élément;
            \item Deuxième élément;
            \item Troisième élément;
        \end{enumerate}
    \end{minipage}
    \end{center}
    \vspace{0.2cm}
\end{exemple}

Si, comme dans ce gabarit, on utilise le package \verb+enumerate+ (commande \verb+\usepackage{enumerate}+ dans le préambule), alors on peut modifier le format de l'énumération avec une option facultative entre [crochets] après la commande \verb+\begin{enumerate}+.

\begin{exemple}
\label{ex:exempleEnumerateFormat}
Le code suivant :
    \begin{center}
    \begin{verbatim}
        Voici une liste ordonnée :
        \begin{enumerate}[(a)]
            \item Premier élément;
            \item Deuxième élément;
            \item Troisième élément;
        \end{enumerate}
    \end{verbatim}
    \end{center}
    
    Produit le résultat suivant :
    \vspace{0.2cm}
    \begin{center}
    \begin{minipage}{0.9\textwidth}
        \setlength{\parindent}{12pt}
        Voici une liste ordonnée :
        \begin{enumerate}[(a)]
            \item Premier élément;
            \item Deuxième élément;
            \item Troisième élément;
        \end{enumerate}
    \end{minipage}
    \end{center}
    \vspace{0.2cm}
\end{exemple}

La spécification du format utilise les caractères spéciaux suivants :
\begin{table}[htb]
    \centering
    \begin{tabular}{ll}
         \verb+1+ & numéros ordinaires.  \\
         \verb+I+ & chiffres romains majuscules. \\
         \verb+i+ & chiffres romains minuscules. \\
         \verb+A+ & lettres majuscules. \\
         \verb+a+ & lettres minuscules.
    \end{tabular}
    \vspace{0.2cm}
    \caption{Tableau des caractères spéciaux du format pour l'environnement \texttt{enumerate}}
    \label{tab:enumerateFormat}
\end{table}

Tous les autres caractères sont réutilisés tels quels. On peut donc faire des listes à puces avec l'environnement \verb+enumerate+, et utiliser le symbole qu'on veut pour les puces :

\begin{exemple}
\label{ex:exempleEnumeratePuces}
Le code suivant :
    \begin{center}
    \begin{verbatim}
        Voici une liste à puces :
        \begin{enumerate}[$\subseteq$]
            \item Premier élément;
            \item Deuxième élément;
            \item Troisième élément;
        \end{enumerate}
    \end{verbatim}
    \end{center}
    
    Produit le résultat suivant :
    \vspace{0.2cm}
    \begin{center}
    \begin{minipage}{0.9\textwidth}
        \setlength{\parindent}{12pt}
        Voici une liste à puces :
        \begin{enumerate}[$\subseteq$]
            \item Premier élément;
            \item Deuxième élément;
            \item Troisième élément;
        \end{enumerate}
    \end{minipage}
    \end{center}
    \vspace{0.2cm}
\end{exemple}

On peut faire des listes dans des listes :

\begin{exemple}
\label{ex:exempleEnumerateAutre}
Le code suivant :
    \begin{center}
    \begin{verbatim}
        Voici une liste à puces :
        \begin{enumerate}[\textbf{ I. }]
            \item Premier élément;
            \item Deuxième élément;
            \item Troisième élément;
            \begin{enumerate}[\textit{ a. }]
                \item Premier sous-élément;
                \item Deuxième sous-élément.
            \end{enumerate}
        \end{enumerate}
    \end{verbatim}
    \end{center}
    
    Produit le résultat suivant :
    \vspace{0.2cm}
    \begin{center}
    \begin{minipage}{0.9\textwidth}
        \setlength{\parindent}{12pt}
        Voici une liste à puces :
        \begin{enumerate}[{\bf  I. }]
            \item Premier élément;
            \item Deuxième élément;
            \item Troisième élément;
            \begin{enumerate}[ -- a --$\circ \Leftrightarrow$]
                \item Premier sous-élément;
                \item Deuxième sous-élément.
            \end{enumerate}
        \end{enumerate}
    \end{minipage}
    \end{center}
    \vspace{0.2cm}
\end{exemple}

\subsection{Les environnements de type \og théorèmes \fg{} }.

Le package \verb+amsthm+ (que nous utilisons, commande \verb+\usepackage{amsthm}+ dans le préambule) permet de créer toutes sortes d'environnements spéciaux pour écrire des énoncés de théorèmes. Comme celui-ci :

\begin{theoreme}
\label{thm:FondamentalDeL'algebre}
    Soit $p(x)$ un polynôme de degré $n$ en une variable complexe $x \in \mathbb C$.
    
    Alors, $p$ possède exactement $n$ racines en comptant les multiplicités.
\end{theoreme}

Ces théorèmes ne sont pas données d'avance, mais dans le préambule du présent gabarit, j'en définis tout plein. Voici un petit tableau qui regroupe les noms des environnements.

\begin{table}[h]
    \centering
    \begin{tabular}{r l | r l}
         \verb+theoreme+, \verb+theoreme*+ & Théorème &
         \verb+proposition+, \verb+proposition*+ & Proposition \\
         \verb+lemme+, \verb+lemme*+ & Lemme &
         \verb+corollaire+, \verb+corollaire*+ & Corollaire \\
         \verb+conjecture+, \verb+conjecture*+ & Conjecture &
         \verb+defnition+, \verb+definition*+ & Définition \\
         \verb+notation+, \verb+notation*+ & Notation &
         \verb+exercice+, \verb+exercice*+ & Exercice \\
         \verb+probleme+, \verb+probleme*+ & Problème &
         \verb+solution+                & Solution \\
         \verb+exemple+, \verb+exemple*+ & Exemple &
         \verb+remarque+, \verb+remarque*+ & Remarque \\
    \end{tabular}
    \vspace{0.2cm}
    \caption{Liste des environnements définis dans le préambule.}
    \label{tab:environnementsPreambule}
\end{table}

Les environnements qui n'ont pas d'astérisque \verb+*+ sont numérotés (sauf \verb+solution+). Les environnements avec un astérisque n'ont pas de numéro.

\begin{exemple}
\label{ex:exempleEnvironnement}
    Le code suivant :
    \begin{center}
    \begin{verbatim}
        \begin{corollaire}
        \label{coro:corollaire1}
            Il suit que tout polynôme $p(x)$ de degré $n$ d'une variable $x \in \mathbb R$ a
            au plus $n$ racines réelles.
        \end{corollaire}
    \end{verbatim}
    \end{center}
    
    Produit le résultat suivant :
    \vspace{0.2cm}
    \begin{center}
    \begin{minipage}{0.9\textwidth}
        \setlength{\parindent}{12pt}
        \begin{corollaire}
        \label{coro:corollaire1}
            Il suit que tout polynôme $p(x)$ de degré $n$ d'une variable $x \in \mathbb R$ a
            au plus $n$ racines réelles.
        \end{corollaire}
    \end{minipage}
    \end{center}
    \vspace{0.2cm}
\end{exemple}

\section{Les raccourcis définis dans ce gabarit}

\LaTeX{} nous permet de définir nos propres commandes. J'en ai défini quelques unes dans
le préambule de ce gabarit, simplement pour nous simplifier un peu la vie. Le tableau suivant montre ces raccourcis :
\begin{table}[htb]
    \centering
    \begin{tabular}{r l l}
        L'ensemble des naturels & \verb+\N+ & $\N$ \\
        L'ensemble des entiers & \verb+\Z+ & $\Z$ \\
        L'ensemble des rationnels & \verb+\Q+ & $\Q$ \\
        L'ensembles des réels & \verb+\R+ & $\R$  \\
        L'ensemble des complexes & \verb+\C+ & $\C$ \\
        Un corps quelconque & \verb+\K+ & $\K$ \\
        \hline \\
        La valeur absolue & \verb+\abs{x}+ & $\abs{x}$ \\
        La cardinalité (pareil comme la valeur absolue) & \verb+\card{x}+ & $\card{x}$ \\
        Les accolades & \verb+\ac{ ... }+ & $\ac{ ... }$ \\
        Les crochets & \verb+\ct{ ... }+ & $\ct{ ... }$ \\
        Les parenthèses & \verb+\pr{ ... }+ & $\pr{...}$ \\
        La norme & \verb+\norm{x}+ & $\norm{x}$ \\
        L'entier inférieur & \verb+\floor{x}+ & $\floor{x}$ \\
        L'entier supérieur & \verb+\ceil{x}+ & $\ceil{x}$ \\
        La barre de conditionnement & \verb+\ac{ A \cond b }+ & $\ac{ A \cond B }$ \\
        \hline \\
        La mesure de probabilités & \verb+\PR+ & $\PR$ \\
        L'espérance & \verb+\ES+ & $\ES$ \\
        La variance & \verb+\Var+ & $\Var$ \\
        La covariance & \verb+\Cov+ & $\Cov$. \\
        \hline \\
        La différece symétrique d'ensembles & \verb+E \dsym F+ & $E \dsym F$ \\
        Une fonction indicatrice & \verb+\1_E (x)+ & $\1_E (x)$
    \end{tabular}
    \vspace{0.2cm}
    \caption{Liste des raccourcis définis dans ce gabarit.}
    \label{tab:raccourcis}
\end{table}

\begin{exemple}
\label{ex:exempleRaccourcis}
    Le code suivant :
    \begin{center}
    \begin{verbatim}
        \begin{theoreme}[Théorème de la limite centrale]
        \label{thm:limites:TLC}
        	Soit $(\Omega, \mathcal E, \PR)$ un espace de probabilités,
        	soit $(X_n : \Omega \to \R)_{n \in \N}$ une suite de variables 
        	aléatoires indépendantes et identiquement distribuées, 
        	avec $\ES \ct{ \abs{X_1}^2} < +\infty$, 
        	$\mu = \ES \ct{X_1}$ et $\sigma^2 = \Var \ct{X_1}$.
        	
        	Soit $S_n = \sum_{k=1}^n X_k$ la somme des $n$ premières 
        	variables aléatoires.
        	
        	Alors, $Z_n = (S_n - n\mu)/\sqrt{n\sigma^2}$ converge 
        	en distribution vers une normale centrée réduite 
        	-- c'est à dire que, pour tout $x \in \R$,
        	
        	\begin{equation}
        	\label{eq:thm:limites:TLC}
        		\lim_{n \to \infty} 
        		    \PR \ac{ 
        		        \frac{S_n - n\mu}{\sqrt{n \sigma^2}} \leq x
        		        } = \Phi (x).
        	\end{equation}
        \end{theoreme}
    \end{verbatim}
    \end{center}
    
    Produit le résultat suivant :
    \vspace{0.2cm}
    \begin{center}
    \begin{minipage}{0.9\textwidth}
        \setlength{\parindent}{12pt}
        \begin{theoreme}[Théorème de la limite centrale]
        \label{thm:limites:TLC}
        	Soit $(\Omega, \mathcal E, \PR)$ un espace de probabilités,
        	soit $(X_n : \Omega \to \R)_{n \in \N}$ une suite de variables 
        	aléatoires indépendantes et identiquement distribuées, 
        	avec $\ES \ct{ \abs{X_1}^2} < +\infty$, 
        	$\mu = \ES \ct{X_1}$ et $\sigma^2 = \Var \ct{X_1}$.
        	
        	Soit $S_n = \sum_{k=1}^n X_k$ la somme des $n$ premières 
        	variables aléatoires.
        	
        	Alors, $Z_n = (S_n - n\mu)/\sqrt{n\sigma^2}$ converge 
        	en distribution vers une normale centrée réduite 
        	-- c'est à dire que, pour tout $x \in \R$,
        	
        	\begin{equation}
        	\label{eq:thm:limites:TLC}
        		\lim_{n \to \infty} 
        		    \PR \ac{ 
        		        \frac{S_n - n\mu}{\sqrt{n \sigma^2}} \leq x
        		        } = \Phi (x).
        	\end{equation}
        \end{theoreme}
    \end{minipage}
    \end{center}
    \vspace{0.2cm}
\end{exemple}

\section{Encore plus}
\label{s:plus}

Il y a encore plus de choses à découvrir avec \LaTeX{} -- mais la base est ici.
\end{document}

